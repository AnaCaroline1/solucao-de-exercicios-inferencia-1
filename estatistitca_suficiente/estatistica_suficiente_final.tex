% Options for packages loaded elsewhere
\PassOptionsToPackage{unicode}{hyperref}
\PassOptionsToPackage{hyphens}{url}
\PassOptionsToPackage{dvipsnames,svgnames,x11names}{xcolor}
%
\documentclass[
  letterpaper,
  DIV=11,
  numbers=noendperiod]{scrartcl}

\usepackage{amsmath,amssymb}
\usepackage{iftex}
\ifPDFTeX
  \usepackage[T1]{fontenc}
  \usepackage[utf8]{inputenc}
  \usepackage{textcomp} % provide euro and other symbols
\else % if luatex or xetex
  \usepackage{unicode-math}
  \defaultfontfeatures{Scale=MatchLowercase}
  \defaultfontfeatures[\rmfamily]{Ligatures=TeX,Scale=1}
\fi
\usepackage{lmodern}
\ifPDFTeX\else  
    % xetex/luatex font selection
\fi
% Use upquote if available, for straight quotes in verbatim environments
\IfFileExists{upquote.sty}{\usepackage{upquote}}{}
\IfFileExists{microtype.sty}{% use microtype if available
  \usepackage[]{microtype}
  \UseMicrotypeSet[protrusion]{basicmath} % disable protrusion for tt fonts
}{}
\makeatletter
\@ifundefined{KOMAClassName}{% if non-KOMA class
  \IfFileExists{parskip.sty}{%
    \usepackage{parskip}
  }{% else
    \setlength{\parindent}{0pt}
    \setlength{\parskip}{6pt plus 2pt minus 1pt}}
}{% if KOMA class
  \KOMAoptions{parskip=half}}
\makeatother
\usepackage{xcolor}
\setlength{\emergencystretch}{3em} % prevent overfull lines
\setcounter{secnumdepth}{-\maxdimen} % remove section numbering
% Make \paragraph and \subparagraph free-standing
\makeatletter
\ifx\paragraph\undefined\else
  \let\oldparagraph\paragraph
  \renewcommand{\paragraph}{
    \@ifstar
      \xxxParagraphStar
      \xxxParagraphNoStar
  }
  \newcommand{\xxxParagraphStar}[1]{\oldparagraph*{#1}\mbox{}}
  \newcommand{\xxxParagraphNoStar}[1]{\oldparagraph{#1}\mbox{}}
\fi
\ifx\subparagraph\undefined\else
  \let\oldsubparagraph\subparagraph
  \renewcommand{\subparagraph}{
    \@ifstar
      \xxxSubParagraphStar
      \xxxSubParagraphNoStar
  }
  \newcommand{\xxxSubParagraphStar}[1]{\oldsubparagraph*{#1}\mbox{}}
  \newcommand{\xxxSubParagraphNoStar}[1]{\oldsubparagraph{#1}\mbox{}}
\fi
\makeatother


\providecommand{\tightlist}{%
  \setlength{\itemsep}{0pt}\setlength{\parskip}{0pt}}\usepackage{longtable,booktabs,array}
\usepackage{calc} % for calculating minipage widths
% Correct order of tables after \paragraph or \subparagraph
\usepackage{etoolbox}
\makeatletter
\patchcmd\longtable{\par}{\if@noskipsec\mbox{}\fi\par}{}{}
\makeatother
% Allow footnotes in longtable head/foot
\IfFileExists{footnotehyper.sty}{\usepackage{footnotehyper}}{\usepackage{footnote}}
\makesavenoteenv{longtable}
\usepackage{graphicx}
\makeatletter
\def\maxwidth{\ifdim\Gin@nat@width>\linewidth\linewidth\else\Gin@nat@width\fi}
\def\maxheight{\ifdim\Gin@nat@height>\textheight\textheight\else\Gin@nat@height\fi}
\makeatother
% Scale images if necessary, so that they will not overflow the page
% margins by default, and it is still possible to overwrite the defaults
% using explicit options in \includegraphics[width, height, ...]{}
\setkeys{Gin}{width=\maxwidth,height=\maxheight,keepaspectratio}
% Set default figure placement to htbp
\makeatletter
\def\fps@figure{htbp}
\makeatother

\KOMAoption{captions}{tableheading}
\usepackage{geometry}            % Controle de margens
\geometry{top=2cm, bottom=2cm, left=3cm, right=3cm}
\makeatletter
\@ifpackageloaded{caption}{}{\usepackage{caption}}
\AtBeginDocument{%
\ifdefined\contentsname
  \renewcommand*\contentsname{Índice}
\else
  \newcommand\contentsname{Índice}
\fi
\ifdefined\listfigurename
  \renewcommand*\listfigurename{Lista de Figuras}
\else
  \newcommand\listfigurename{Lista de Figuras}
\fi
\ifdefined\listtablename
  \renewcommand*\listtablename{Lista de Tabelas}
\else
  \newcommand\listtablename{Lista de Tabelas}
\fi
\ifdefined\figurename
  \renewcommand*\figurename{Figura}
\else
  \newcommand\figurename{Figura}
\fi
\ifdefined\tablename
  \renewcommand*\tablename{Tabela}
\else
  \newcommand\tablename{Tabela}
\fi
}
\@ifpackageloaded{float}{}{\usepackage{float}}
\floatstyle{ruled}
\@ifundefined{c@chapter}{\newfloat{codelisting}{h}{lop}}{\newfloat{codelisting}{h}{lop}[chapter]}
\floatname{codelisting}{Listagem}
\newcommand*\listoflistings{\listof{codelisting}{Lista de Listagens}}
\makeatother
\makeatletter
\makeatother
\makeatletter
\@ifpackageloaded{caption}{}{\usepackage{caption}}
\@ifpackageloaded{subcaption}{}{\usepackage{subcaption}}
\makeatother

\ifLuaTeX
\usepackage[bidi=basic]{babel}
\else
\usepackage[bidi=default]{babel}
\fi
\babelprovide[main,import]{portuguese}
% get rid of language-specific shorthands (see #6817):
\let\LanguageShortHands\languageshorthands
\def\languageshorthands#1{}
\ifLuaTeX
  \usepackage{selnolig}  % disable illegal ligatures
\fi
\usepackage{bookmark}

\IfFileExists{xurl.sty}{\usepackage{xurl}}{} % add URL line breaks if available
\urlstyle{same} % disable monospaced font for URLs
\hypersetup{
  pdftitle={Estimadores Suficientes},
  pdfauthor={Ana Caroline Alexandre P.},
  pdflang={pt},
  colorlinks=true,
  linkcolor={blue},
  filecolor={Maroon},
  citecolor={Blue},
  urlcolor={Blue},
  pdfcreator={LaTeX via pandoc}}


\title{Estimadores Suficientes}
\usepackage{etoolbox}
\makeatletter
\providecommand{\subtitle}[1]{% add subtitle to \maketitle
  \apptocmd{\@title}{\par {\large #1 \par}}{}{}
}
\makeatother
\subtitle{Solução de exemplos e exercícios}
\author{Ana Caroline Alexandre P.}
\date{2025-01-17}

\begin{document}
\maketitle


\subsubsection{Exercício 11 :}\label{exercuxedcio-11}

Sejam \(X_1,..., X_n\) uma amostra aleatória da distribuição de Poisson
com parâmetro \(θ\). Vamos verificar se \(T = \sum_{i=1}^{n} X_i\) é
suficiente para \(\theta\).

\subsubsection{Solução 11:}\label{soluuxe7uxe3o-11}

Temos pelo Critério da Fatoração que,

\[
L(\theta|x)= h(x_1,...,x_n)\times g_\theta[T(x_1,...,x_n)]
\]

A função de probabilidade da Poisson é

\[
P(X=x) = \frac{e^{-\theta}\theta^x}{x!} , \quad x=0,1,2,3...
\]

A função de verossimilhança aplicada a Poisson será:

\[L(\theta|x)=\prod_{i=1}^{n} P(X = x_i) = \prod_{i=1}^{n} \frac{e^{-\theta}\theta^{x_i}}{x_i!}
\\=\frac{e^{-\theta}\theta^{x_1}}{x_1!}\times\frac{e^{-\theta}\theta^{x_2}}{x_2!},...,\frac{e^{-\theta}\theta^{x_n}}{x_n!}=
\]
\[=\frac{e^{-n\theta}\times\theta^{\sum_{i=1}^{n}x_i}}{\prod_{i=1}^{n}x_i!}=\\
\frac{1}{\prod_{i=1}^{n}x_i!}\times{e^{-n\theta}\theta^{\sum_{i=1}^{n}x_i}}
\]

\begin{enumerate}
\def\labelenumi{\arabic{enumi}.}
\item
  Onde \(\frac{1}{\prod_{i=1}^{n}x_i!}\) é função que depende apenas da
  amostra \(h(x_1,...x_n)\).
\item
  E \({e^{-n\theta}\theta^{\sum_{i=1}^{n}x_i}}\) é a função \(g_\theta\)
  que depende de \(\theta\) e de \(x_i\) somente através da
  \emph{estatística T}.
\end{enumerate}

Nota-se que \(T(x_1,x_2,...,x_n)=\sum_{i = 1}^{n} x_{i}\). Logo,
\(\sum_{i = 1}^{n} x_{i}\) é uma estatística suficiente para \(\theta\).

\subsubsection{Exercício (aplicação em sala)
1:}\label{exercuxedcio-aplicauxe7uxe3o-em-sala-1}

Sendo \(X\) \textasciitilde{} \(Ber(\theta)\), com: \[
P(X=x| \theta)=\theta^x(1-\theta)^{1-x}, \quad x = 0,1,2,3...
\]

\subsubsection{Solução:}\label{soluuxe7uxe3o}

A função de verossimilhança aplicada a Bernoulli: \[
L(\theta|x) = \prod_{i=1}^{n} \theta^{xi}(1-\theta)^{1-x_i} = \theta^{x_1}(1-\theta)^{1-x_1}\times\theta^{x_2}(1-\theta)^{1-x_2},...,\theta^{x_n}(1-\theta)^{1-x_n} =
\]

\[
= \theta^{\sum_{i=1}^{n}x_i}(1-\theta)^{n-\sum_{i=1}^{n}x_i}
\]

\begin{enumerate}
\def\labelenumi{\arabic{enumi}.}
\tightlist
\item
  Temos que a função \(h(x_1,..x_n) = 1\).
\item
  E a
  \(g_\theta[T(x_1,x_2,...,x_n)]= \theta^{\sum_{i=1}^{n}x_i}(1-\theta)^{n-\sum_{i=1}^{n}x_i}\).
\end{enumerate}

Podemos reescrever da seguinte forma: \[
\theta^{\sum_{i=1}^{n}x_i} \times \frac{(1-\theta)^{n}}{(1-\theta)^{\sum_{i=1}^{n}x_i}}
\] Logo \(T=\sum{x_i}\) é uma estatística suficiente para \(\theta\).

\subsubsection{Exercício 12:}\label{exercuxedcio-12}

Sejam \(X_1,...,X_n\) uma amostra aleatória da variável aleatória
\(X ∼ U(0, θ)\).

A função densidade da Uniforme contínua é:

\[
f(x|\theta)= \frac{1}{\theta-0}=\frac{1}{\theta}I_{[0,\theta]}x
\]

\begin{enumerate}
\def\labelenumi{\arabic{enumi}.}
\tightlist
\item
  \(\quad x \in [0,\theta]\), aqui temos um parâmetro no suporte de
  \(x\). A onde não se pode calcular a equação de regulariadade do
  \(Li\).
\end{enumerate}

\subsubsection{Solução 12:}\label{soluuxe7uxe3o-12}

Aplicado a função de verossimilhança:

\[
L(\theta|x) = \prod_{i=1}^{n} f(x_i) = \prod_{i=1}^{n} \frac{1}{\theta} \cdot I(x_i)_{[0,\theta]}= 
\]

\[
=  \frac{1}{\theta} \cdot I(x_1)_{[0,\theta]} \times \frac{1}{\theta} \cdot I(x_2)_{[0,\theta]},...,\frac{1}{\theta} \cdot I(x_n)_{[0,\theta]} =
\]

\[
= \frac{1}{\theta^{n}} \cdot \prod_{i=1}^{n} I(x_i)_{[0,\theta]}
\]

\begin{itemize}
\item
  Quem seria a estatística suficiente?

  \(\rightarrow\) Existe uma relação de dependencia de \(\theta\) no
  suporte de \(x_i\).

  \(\rightarrow\) Para encontrar a estatística, teremos que verificar os
  intervalos do suporte.
\end{itemize}

\textbf{Observe:}

\[
0 < x_i < \theta, \quad i=1,...n.
\]

Logo, podemos expandir da seguinte forma:

\[
0 < x_1 < \theta
\]

\[
0 < x_2 < \theta
\] \[
.
\] \[
.
\] \[
.
\] \[
0 < x_n < \theta
\]

Podemos ter a seguinte relação, baseado nos conceitos anteriores sobre
as \textbf{Estatísticas de Ordem(mínimos, máximos)}:

\[
0 <x_{(1)} < \underbrace{x_{(n)}<\theta<\infty}_{intervalo \quad de \quad  \theta}
\]

Reescrevendo a função Indicadora para o intervalo de \(\theta\), temos:

\[
L(\theta|x) = \frac{1}{\theta^{n}} \cdot \prod_{i=1}^{n} I(x_i)_{[x_{n},\infty]} =
\]

\[
= \frac{1}{\theta^{n}} \cdot I(\theta)_{[x_{(n)}, \infty)}
\] Pelo Critério da Fatoração, temos:

\begin{enumerate}
\def\labelenumi{\arabic{enumi}.}
\tightlist
\item
  Temos que a função \(h(x_1,..x_n) = 1(x_{(1)})_{[0,x_{(n)}]}\).
\item
  E a
  \(g_\theta[T(x_1,x_2,...,x_n)]= \frac{1}{\theta^{n}} \cdot I(\theta)_{[x_{(n)}, \infty)}\).
\end{enumerate}

A partir da função indicadora de uma variável \(x\), ao manipular os
intervalos, podemos obter a função de \(\theta\), de modo que
\(g_{\theta}\) seja uma estatística suficiente.

Assim, a estatística suficiente \(T(x_{1},...,x_{n})= x_{(n)}\) é dada
pelo máximo \(x_{(n)}\), que depende de \(\theta\).

\subsubsection{Exercício (aplicação em sala)
2:}\label{exercuxedcio-aplicauxe7uxe3o-em-sala-2}

Sendo \(X ∼ U(\theta,1)\).

\[
f(x)=\frac{1}{1-\theta} \cdot I(x)_{[\theta,1]}
\]

\subsubsection{Solução:}\label{soluuxe7uxe3o-1}

Aplicando a função de verossimilhança:

\[
L(\theta|x) = \prod_{i=1}^{n} \frac{1}{1-\theta} \cdot I(x_i)_{[\theta,1]}= 
\]

\[
= \frac{1}{1-\theta} \cdot I(x_1)_{[\theta,1]} \times \frac{1}{1-\theta} \cdot I(x_2)_{[\theta,1]} \times,..., \times \frac{1}{1-\theta} \cdot I(x_n)_{[\theta,1]}=   
\]

\[
= \frac{1}{(1-\theta)^{n}} \cdot \prod_{i=1}^{n} I(x_i)_{[\theta,1]}
\]

Novamente, temos que:

\[
0 < x_i < \theta, \quad i=1,...n.
\]

\[
\theta < x_1 < 1
\] \[
\theta < x_2 < 1
\] \[
.
\] \[
.
\] \[
.
\] \[
\theta < x_n < 1
\] Logo, o intervalo em que \(\theta\) está contido, será:

\[
\underbrace{0< \theta < x_{(1)}}_{intervalo \quad de \quad \theta}< x_{(n)} < 1
\]

Reescrevendo a função indicadora para \(\theta\):

\[
L(\theta|x) = \prod_{i=1}^{n} \frac{1}{1-\theta} \cdot I(x_i)_{[\theta,1]}= 
\]

\[
= \frac{1}{(1-\theta)^{n}} \cdot I(\theta)_{[0,x_{(1)}]} \cdot I(x_{(n)})_{[x_{(1)}, 1]}
\]

\begin{enumerate}
\def\labelenumi{\arabic{enumi}.}
\tightlist
\item
  Temos que a função \(h(x_1,..x_n) = 1I(x_{(n)})_{[x_{(1)}, 1]}\).
\item
  E a
  \(g_\theta[T(x_1,x_2,...,x_n)] = \frac{1}{(1-\theta)^{n}} \cdot I(\theta)_{[0,x_{(1)}]}\).
\end{enumerate}

Portanto, pelo Critério da Fatoração, \(T(x_{1},...,x_{n})= x_{(1)}\) é
uma estatística suficiente para \(\theta\).




\end{document}
